\documentclass[12pt,a4paper]{article}
\usepackage[utf8]{inputenc}
\usepackage[german]{babel}
\usepackage[T1]{fontenc}
\usepackage{amsmath}
\usepackage{amsfonts}
\usepackage{amssymb}
\usepackage{graphicx}
\author{Andreas Mueller}
\title{Milestones MBC-Ping-Pong}
\begin{document}

\maketitle
\newpage
\tableofcontents
\newpage

\section{Meilensteine}
\subsection{25.11.2016 Projekt Aufsetzen}
Komponenten wie Node.JS github, npm sowie simple Standartanwendung auf Node.js.

\subsection{16.12.2016 Prototyp (Technik)}
Kommunikationsaufbau WebRTC: Smartphone -> Monitor, node.JS Server als Mittelsmann. (ICE-Server?!)

Positionsdaten vom Smartphone abgreifen.
Positionsdaten von Frontend (Smartphone) an Backend (Monitor) senden.
Frontend (Monitor) zeigt Kugel in Canvas (eventuell inkl. WebGL) gemäß empfangenen positionsdaten (x, y).
Es können sich 2 Personen (Spieler) mit der selben Spielsession verbinden.

REST für Spielaufruf, webRTC für andauernde Kommunikation.
Node.JS Server für static content,
sowie Kommunikationsaufbau für WebRTC.

github.com/nplab/WebRTC-Data-Channel-Playground

\subsection{06.01.2017 Zwei Spieler}
Anzeige der Spielobjekte auf Client-Seite.
Übertragung relevanter Positionsdaten.
Spieler soll spielen können.
Daher: Spielbar!

\subsection{06.01.2017 Angestrebtes Release 1.0}
Release der Version 1.0.
Senden an Martin.
Spielen in Live-Umgebung.

\subsection{bis 24.02.2017 Diverse Features TBD}
Austrittswinkel abhängig von trefferposition auf Balken abhängig.
Drall auf Ball bei bewegtem Balken bei Treffer.

\subsection{24.02.2017 Definitives Release 1.x}
Release der Version 1.x.
Inklusive aller bis hier implementiereten Zusatz-Features.




\end{document}
