\chapter{Frontend}
\section{Auswahl einer Physics-Engine}

\begin{quote}
\href{https://github.com/Transport-Protocol/MBC-Ping-Pong/issues/3}{Auswahl einer geeigneten 2D-/Physiksengine \#3}
\newline
Die Darstellung auf dem Anzeigegerät ist über den DOM nicht möglich, da die Bewegung des Balls und der Schläger zu schnell werden. Zudem wird auch die Physik auf dem Anzeigegeräts berechnet. Hierfür ist eine Kollisionserkennung erforderlich.
\end{quote}
Für die physikalisch korrekte Kollision des Balles mit der Spielwelt und den Schlägern haben wir uns entschieden eine Physics-Engine zu verwenden.

\subsection{Matter.JS}
Matter js ist eine sehr mächtige Engine. Sie unterstützt viele Formen und Physikalische Eigenschaften wie zum Beispiel Masse und wirkende Kräfte auf die jeweiligen Objekte. 
Es besteht die Möglichkeit physikalische Objekte zusammen zusetzen und sogar diese Elastisch erscheinen zu lassen, so ist es beispielsweise möglich Stoff oder schwingende Seile zu erstellen.
Nach mehrstündiger Einarbeitung kam ich zu dem Schluss das diese Engine für unser Projekt nicht geeignet ist.
Gründe hierfür sind:
\begin{itemize}
\item
Zu Komplex:
Die Einrichtung des Spielfeldes erwies sich als überaus schwierig. Gute Anleitungen für einfache Szenarien fehlten, die verfügbaren Anleitungen sind zu grundlegend beschrieben.
Ebenfalls half die Anleitung der Engine nicht bei der Verwendung der einzelnen Komponenten.
\item
Die Positionierung der Objekte bezog sich immer auf den Mittelpunkt des Objektes, man kann beispielsweise kein Rechteck von (x,y) nach (x1,y1) erstellen sondern muss den Mittelpunkt und die Abmaße des Objektes angeben.
\item
Physik nicht immer korrekt. Die Engine sollte den Ball richtig von einer Ebene abprallen lassen, diese Engine allerdings lies den Ball teilweise an Plattformen abrollen obwohl keine Schwerkraft vorhanden war. Ich schließe darauf das die Engine Reibungskräfte und vielleicht sogar Anziehungskräfte zwischen den Objekten herstellt. Für unser Projekt ist dies aber nicht zu gebrauchen.
\item
Schwer zu debuggen. Während meiner Versuche bin ich immer wieder auf Probleme gestoßen. Einige der Debugausgaben ließen sich gut ableiten und waren hilfreich. 
Allerdings bin ich auch auf einige Probleme gestoßen welche nicht in den Debugausgaben behandelt wurde. Meine letzten Versuche endeten alle darin das der Browser gecrasht ist aufgrund eines Memory-leaks.
\end{itemize}
\subsection{Phaser.io}
Phaser.io beschreibt sich selber als html5 Game Framework. Es wurde nach dem mobile-first Prinzip entwickelt und ist opensource.
